\section{Abstract Force}

\subsection{Abstract}

\subsection{Abstract Internal}

\section{Structural Forces}

\subsection{Force}

\subsubsection{Absolute}

\subsubsection{Follower}

\subsubsection{Absolute Internal}

\subsubsection{Follower Internal}

\subsection{Couple}

\subsubsection{Absolute}

\subsubsection{Follower}

\subsubsection{Absolute Internal}

\subsubsection{Follower Internal}

\section{Modal}

\section{External}
The external structural force element allows MBDyn
to cooperate with external solvers by defining a meta-element
that applies forces and moments to a set of structural nodes.
The forces and moments are provided by an external solver,
called the \emph{peer}.
MBDyn provides the peer information about the motion of the nodes
participating in the set.

Optionally, an additional layer of field mapping can be interposed.
In this case, the motion of the nodes is transformed in the motion
of a set of intermediate points, which is further mapped
into the motion of the points known by the peer by means
of a linear mapping.

The forces at the mapped points returned by the peer
are mapped back into forces and moments for the structural nodes
participating in the set of the element.

\subsection{External Structural}
The external structural and external structural mapping forces
can be formulated directly in the absolute frame, or referred
to a reference node.
In the former case, operations are straightforward;
in the latter one, the kinematics are first expressed
in the reference frame of the reference node and then sent
to the peer along with the motion of the reference node.
The latter returns nodal forces and moments oriented according
to the reference frame of the reference node.
The additional operations performed by the mapping variant
are discussed separately in a subsequent section.
The orientation of the reference node is $\TT{R}_r$;
the position is $\T{x}_r$.

The orientation of the generic node $i$ is
\begin{align}
	\TT{R}_i
	&=
	\TT{R}_r \overline{\TT{R}}_i
	.
\end{align}
The relative orientation passed to the peer solver is
\begin{align}
	\overline{\TT{R}}_i
	&=
	\TT{R}_r^T \TT{R}_i
	.
\end{align}

The position of the generic node $i$ is
\begin{align}
	\T{x}_i
	&=
	\T{x}_r
	+
	\TT{R}_r \overline{\T{x}}_i
	.
\end{align}
The relative position passed to the peer solver is
\begin{align}
	\overline{\T{x}}_i
	&=
	\TT{R}_r^T \plbr{\T{x}_i - \T{x}_r}
	.
\end{align}

The angular velocity of the generic node $i$ is
\begin{align}
	\T{\omega}_i\times{}
	&=
	\dot{\TT{R}}_i \TT{R}_i^T
	=
	\dot{\TT{R}}_r \TT{R}_r^T
	+
	\TT{R}_r \dot{\overline{\TT{R}}}_i \overline{\TT{R}}_i^T \TT{R}_r^T
	=
	\T{\omega}_r\times{}
	+
	\TT{R}_r \overline{\T{\omega}}_i\times \TT{R}_r^T
	.
\end{align}
The relative angular velocity passed to the peer solver is
\begin{align}
	\overline{\T{\omega}}_i
	&=
	\TT{R}_r^T \plbr{\T{\omega}_i - \T{\omega}_r}
	.
\end{align}


The velocity of the generic node $i$ is
\begin{align}
	\dot{\T{x}}_i
	&=
	\dot{\T{x}}_r
	+
	\T{\omega}_r \times \TT{R}_r \overline{\T{x}}_i
	+
	\TT{R}_r \dot{\overline{\T{x}}}_i
	.
\end{align}
The relative velocity passed to the peer solver is
\begin{align}
	\dot{\overline{\T{x}}}_i
	&=
	\TT{R}_r^T \plbr{
		\dot{\T{x}}_i
		-
		\dot{\T{x}}_r
	}
	-
	\plbr{\TT{R}_r^T \T{\omega}_r} \times \overline{\T{x}}_i
\end{align}

\textbf{TODO: accelerations}

The virtual work done by the forces applied to the nodes is
\begin{align}
	\delta\mathcal{L}
	&=
	\delta\T{x}_r \cdot \T{f}_r
	+
	\T{\theta}_{r\delta} \cdot \T{m}_r
	+
	\sum_i \plbr{
		\delta\overline{\T{x}}_i \cdot \overline{\T{f}}_i
		+
		\overline{\T{\theta}}_{i\delta} \cdot \overline{\T{m}}_i
	}
	.
\end{align}

The virtual rotation of the generic node $i$ is
\begin{align}
	\delta\TT{R}_i \TT{R}_i^T
	&=
	\T{\theta}_{i\delta}\times{}
	=
	\delta\TT{R}_r \TT{R}_r^T
	+
	\TT{R}_r \delta\overline{\TT{R}}_i \overline{\TT{R}}_i^T \TT{R}_r^T
	=
	\T{\theta}_{r\delta}\times{}
	+
	\TT{R}_r \overline{\T{\theta}}_{i\delta}\times \TT{R}_r^T
	.
\end{align}
The relative virtual rotation is thus
\begin{align}
	\overline{\T{\theta}}_{i\delta}
	&=
	\TT{R}_r^T \plbr{\T{\theta}_{i\delta} - \T{\theta}_{r\delta}}
	.
\end{align}

The virtual displacement of the generic node $i$ is
\begin{align}
	\delta\T{x}_i
	&=
	\delta\T{x}_r
	+
	\T{\theta}_{r\delta}\times \TT{R}_r \overline{\T{x}}_i
	+
	\TT{R}_r\delta\overline{\T{x}}_i
	.
\end{align}
The relative virtual displacement is thus
\begin{align}
	\delta\overline{\T{x}}_i
	&=
	\TT{R}_r^T \plbr{
		\delta\T{x}_i
		-
		\delta\T{x}_r
	}
	+
	\overline{\T{x}}_i \times \plbr{\TT{R}_r^T \T{\theta}_{r\delta}}
	.
\end{align}

The virtual work becomes
\begin{align}
	\delta\mathcal{L}
	&=
	\delta\T{x}_r \cdot \T{f}_r
	+
	\T{\theta}_{r\delta} \cdot \T{m}_r
	+
	\sum_i \plbr{
		\delta\T{x}_i \cdot \TT{R}_r \overline{\T{f}}_i
		+
		\T{\theta}_{i\delta} \cdot \TT{R}_r \overline{\T{m}}_i
		-
		\delta\T{x}_r \cdot \TT{R}_r \overline{\T{f}}_i
		-
		\T{\theta}_{r\delta} \cdot \TT{R}_r \plbr{
			\overline{\T{m}}_i
			+
			\overline{\T{x}}_i \times \overline{\T{f}}_i
		}
	}
	.
\end{align}
The force and moment acting on the generic node $i$ are
\begin{align}
	\T{f}_i
	&=
	\TT{R}_r \overline{\T{f}}_i
	\\
	\T{m}_i
	&=
	\TT{R}_r \overline{\T{m}}_i
	.
\end{align}
The moment is always intrinsically referred to the current position
of the node.
The force and moment acting on the reference node are
\begin{align}
	\T{f}
	&=
	\T{f}_r - \sum_i \TT{R}_r \overline{\T{f}}_i
	\\
	\T{m}
	&=
	\T{m}_r - \sum_i \TT{R}_r \plbr{
		\overline{\T{m}}_i
		+
		\overline{\T{x}}_i \times \overline{\T{f}}_i
	}
	.
\end{align}
The moment is always intrinsically referred to the current position
of the node.

In principle, $\T{f}$ and $\T{m}$ should be identically zero,
unless the reference node is specifically loaded.
This fact can be exploited by setting \texttt{use reference node forces}
to \texttt{no}, which results in ignoring $\T{f}$ and $\T{m}$.



\subsection{External Structural Mapping}
The external structural mapping case differs from the previous one
in the fact that two additional intermediate layers are added.
The first layer computes the motion of a set of points
rigidly connected to the nodes that participate in the set.
Each node $n$ can have an arbitrary number of associated points $p$,
whose position in the reference frame of the node is $\tilde{\T{o}}_{np}$.
The position of point $p$ associated to node $n$ is
\begin{align}
	\T{x}_{np}
	&=
	\T{x}_n
	+
	\TT{R}_n \tilde{\T{o}}_{np}
	\label{eq:forces:external-mapping:x_np}
	.
\end{align}
The position $\T{x}_{np}$, instead of $\T{x}_i$, must be used to compute
the relative position.
Orientations are not mapped.

As soon as the positions, velocities (and accelerations, if needed)
of the points are computed and collected in vectors
denoted by the subscript $(\cdot)_\text{mbdyn}$,
they are mapped into the corresponding quantities
of the peer, denoted by the subscript $(\cdot)_\text{peer}$,
using a linear transformation $\TT{H}$, namely
\begin{align}
	\T{x}_\text{peer}
	&=
	\TT{H} \T{x}_\text{mbdyn}
	\\
	\dot{\T{x}}_\text{peer}
	&=
	\TT{H} \dot{\T{x}}_\text{mbdyn}
	\\
	\delta\T{x}_\text{peer}
	&=
	\TT{H} \delta\T{x}_\text{mbdyn}
	.
\end{align}
The work done in the peer domain must be equal by the work done
in MBDyn's; this implies
\begin{align}
	\delta\mathcal{L}_\text{mbdyn}
	&=
	\delta\T{x}_\text{mbdyn} \cdot \T{f}_\text{mbdyn}
	=
	\delta\T{x}_\text{peer} \cdot \T{f}_\text{peer}
	=
	\delta\mathcal{L}_\text{peer}
	,
\end{align}
which yields
\begin{align}
	\delta\T{x}_\text{mbdyn} \cdot \T{f}_\text{mbdyn}
	&=
	\delta\T{x}_\text{mbdyn} \cdot \TT{H}^T \T{f}_\text{peer}
	,
\end{align}
and thus, thanks to the arbitrariness of virtual displacements,
\begin{align}
	\T{f}_\text{mbdyn}
	&=
	\TT{H}^T \T{f}_\text{peer}
	.
\end{align}
The forces $\T{f}_\text{mbdyn}$ corrrespond to the points
of the intermediate mapping layer.
They are transformed into the corresponding nodal forces and moments
considering the work done by the virtual perturbation
of Eq.~(\ref{eq:forces:external-mapping:x_np}),
\begin{align}
	\delta\T{x}_{np}
	&=
	\delta\T{x}_n
	-
	\plbr{\TT{R}_n \tilde{\T{o}}_{np}} \times \T{\theta}_{n\delta}
	,
\end{align}
namely
\begin{align}
	\delta\mathcal{L}_n
	&=
	\sum_p \delta\T{x}_{np} \cdot \T{f}_{np}
	=
	\sum_p \plbr{
		\delta\T{x}_n \cdot \T{f}_{np}
		+
		\T{\theta}_{n\delta} \cdot \plbr{\TT{R}_n \tilde{\T{o}}_{np}} \times \T{f}_{np}
	}
	.
\end{align}
The corresponding nodal force and moment are
\begin{align}
	\T{f}_n
	&=
	\sum_p \T{f}_{np}
	\\
	\T{m}_n
	&=
	\sum_p \plbr{\TT{R}_n \tilde{\T{o}}_{np}} \times \T{f}_{np}
	.
\end{align}
The moment is always intrinsically referred to the current position
of the node.

When a reference node is defined, all symbols in the expressions
of the nodal force and moment must bear an overbar $\overline{(\cdot)}$,
to indicate that they are relative and thus need
to be further transformed as previously shown
for the external structural force element.

\subsection{External Modal}

\subsection{External Modal Mapping}

\subsection{Client Library}

The peer side of the communication protocol has been implemented
in a set of client libraries.
At the core there is \texttt{libmbc}, a library written in C
that performs the core operations.
Declarations are provided in the header file \texttt{mbc.h}.

A high-level interface is provided in C++ in \texttt{libmbcxx}.
Declarations are provided in the header file \texttt{mbcxx.h}.

Another interface is provided in Python, in module \texttt{\_mbc\_py.so}.
The corresponding Python code is defined in \texttt{mbc\_py\_interface.py}.

\subsubsection{Socket-Based Protocol}
Each communication is prefixed by a \texttt{uint8\_t} value
that indicates the type of operation being performed.
It is optionally followed by a message.
Legal values are
\begin{itemize}
\item \texttt{ES\_REGULAR\_DATA}
\item \texttt{ES\_GOTO\_NEXT\_STEP}
\item \texttt{ES\_ABORT}
\item \texttt{ES\_REGULAR\_DATA\_AND\_GOTO\_NEXT\_STEP}
\item \texttt{ES\_NEGOTIATION}
\item \texttt{ES\_OK}
\end{itemize}
The corresponding numerical value is defined in \texttt{mbc.h}.

Operations are:
\begin{itemize}
\item negotiation: the client tells the server what
\item kinematics exchange: the master sends the motion
\item forces exchange: the peer receives the loads
\end{itemize}

