\documentclass{article}

\usepackage{amsmath,amssymb,bm}
\usepackage{url}

\begin{document}

\title{The Muscles Module}
\author{Pierangelo Masarati and Andrea Zanoni}
\date{}
\maketitle

\begin{abstract}
This document describes the implementation of the \texttt{module-muscles}
run-time loadable module for MBDyn (\url{http://www.mbdyn.org/}).
\end{abstract}

\section{Introduction}
\cite{PENNESTRI-2007-JB},
\cite{ZANONI-2012-IMSD},
\cite{ZANONI-2012-IMECE}



\section{Formulation}



\section{Implementation}
The muscle model is implemented as a run-time loadable constitutive law
intended to be used with the rod joint.
The rod joint computes the axial strain $\varepsilon$
from the distance $l$ between the two reference points it connects as
\begin{align}
	\varepsilon
	&=
	\frac{l}{l_i} - 1
\end{align}
where $l_i$ is the initial length of the rod.
The rod also computes the strain rate as
\begin{align}
	\dot{\varepsilon}
	&=
	\frac{\dot{l}}{l_i}
\end{align}
The rod passes the strain and the strain rate to the constitutive law
using the \texttt{Update()} member function.

Subsequently, the rod joint can retrieve from the constitutive law
the force $f$ and the equivalent stiffness $f_{/\varepsilon}$
and damping $f_{/\dot{\varepsilon}}$ using specific methods.

The muscles' constitutive law computes those parameters
within the \texttt{Update()} member function.
The muscle's non-dimensional elongation and elongation rate
are computed from the strain and the strain rate as
\begin{subequations}
\begin{align}
	x
	&=
	\frac{l}{l_0}
	=
	\frac{l_i}{l_0}(1 + \varepsilon)
	\\
	v
	&=
	\frac{\dot{l}}{V_0}
	=
	\frac{l_i}{V_0} \dot{\varepsilon}
\end{align}
\end{subequations}
The equivalent stiffness and damping are computed as
\begin{subequations}
\begin{align}
	f_{/\varepsilon}
	&=
	f_{m/x} x_{\varepsilon}
	=
	f_{m/x} \frac{l_i}{l_0}
	\\
	f_{/\dot{\varepsilon}}
	&=
	f_{m/v} v_{/\dot{\varepsilon}}
	=
	f_{m/v} \frac{l_i}{V_0}
\end{align}
\end{subequations}
The activation $a$ is the result of evaluating a ``driver'';
as such, the activation can be modified during the analysis.

\paragraph{Ergonomy.}
When the ``ergonomy'' variant is used, $f_{m/v} \equiv 0$.

\paragraph{Reflexive.}
When the ``reflexive'' variant is used, the actual activation
is computed as
\begin{align}
	a
	&=
	a_\text{ref}
	+
	K_p \left( x - \frac{l_\text{ref}}{l_0} \right)
	+
	K_d \left( v - \frac{\dot{l}_\text{ref}}{V_0} \right)
\end{align}
While $l_\text{ref}$ is the result of evaluating a specific driver,
$\dot{l}_\text{ref}$ is always zero in the current implementation.

The force is computed using the actual value of $a$,
while the equivalent stiffness and damping are computed as
\begin{subequations}
\begin{align}
	f_{/\varepsilon}
	&=
	f_{/\varepsilon}(a_\text{ref})
	+
	f_{m/a} K_p x_{/\varepsilon}
	=
	f_{/\varepsilon}(a_\text{ref})
	+
	f_{m/a} K_p \frac{l_i}{l_0}
	\\
	f_{/\dot{\varepsilon}}
	&=
	f_{/\dot{\varepsilon}}(a_\text{ref})
	+
	f_{m/a} K_d v_{/\dot{\varepsilon}}
	=
	f_{/\dot{\varepsilon}}(a_\text{ref})
	+
	f_{m/a} K_d \frac{l_i}{V_0}
\end{align}
\end{subequations}


\section{Concluding Remarks}


\bibliographystyle{unsrt}
\bibliography{../../manual/mybib}

\end{document}
